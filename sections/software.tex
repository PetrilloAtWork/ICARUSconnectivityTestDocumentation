\section{Supporting library software}
\label{app:supporting-library-software}

The analysis is supported by a utility library which is available as GIT
repository, with the main repository being stored by GitHub at
\texttt{github.com:PetrilloAtWork/ICARUSconnectivityTest.git}. It is
possible to download it with:

\begin{verbatim}
git clone git@github.com:PetrilloAtWork/ICARUSconnectivityTest.git
\end{verbatim}

or equivalent.\\

The analysis library used for the online analysis is tagged as
\texttt{v03\_00}. In that version, both waveform parsing and analysis
code are in \texttt{drawWaveforms.py}.\\

Utility classes are provided:

\begin{itemize}
\tightlist
\item
  \texttt{WaveformSourceInfo}: a data structure describing a waveform by
  its parameters: chimney, connection, test box switch position,
  oscilloscope channel and index

  \begin{itemize}
  \tightlist
  \item
    \texttt{channel} attribute is provided with the number of connection
    channel, computed from the switch position and oscilloscope channel
    and ranging between \texttt{1} and \texttt{32}
  \item
    \texttt{increaseIndex()} is a shortcut to increase the index
    attribute
  \item
    \texttt{firstIndexOf(position)} is a static method returning the
    index of the first waveform on a given switch position
  \end{itemize}
\item
  \texttt{WaveformSourceParser}: a class that represents the group of
  waveform files which contains a specified one, the ``triggering
  file'':

  \begin{itemize}
  \tightlist
  \item
    the triggering file is specified at construction or with
    \texttt{parse()} method
  \item
    the parameters of the triggering files are accessible via
    \texttt{sourceInfo} (a \texttt{WaveformSourceInfo} instance)
  \item
    the pattern of the file path is stored as
    \texttt{sourceFilePattern}, and a full file name can be obtained by
    \texttt{sourceFilePattern\ \%\ sourceInfo} (with \texttt{sourceInfo}
    \emph{any} valid \texttt{WaveformSourceInfo})
  \item
    the list of expected names for all \texttt{N} files at the current
    (or the specified) oscilloscope channel can be obtained via
    \texttt{allChannelSources()}
  \item
    the list of expected names for all \texttt{4N} files at the current
    test box switch position can be obtained via
    \texttt{allPositionSources()}
  \end{itemize}
\end{itemize}

Plotting functions:

\begin{itemize}
\tightlist
\item
  \texttt{plotWaveformFromFile()} creates and returns a single
  \texttt{TGraph} with the specified waveforms
\item
  \texttt{plotAllPositionWaveforms()} returns a \texttt{TCanvas} split
  in as many pads as there are oscilloscope channels (\texttt{4}), and
  plots in each pad all the waveforms from a channel

  \begin{itemize}
  \tightlist
  \item
    the first argument is a \texttt{WaveformSourceParser} object which
    determines which waveforms are plotted (equivalent to
    \texttt{allPositionSources()})
  \item
    the vertical range of all graphs is fixed to be at least between 1
    and 3 volt
  \item
    statistics are extracted and printed in each pad, as documented
    below
  \end{itemize}
\item
  \texttt{plotAllPositionsAroundFile()} plots all the waveforms for the
  same connector and test box switch position as the one of the
  specified waveform file; the plots are described with
  \texttt{plotAllPositionWaveforms()} above
\end{itemize}

The statistics box of the plots includes: * ``waveforms'': the number
\texttt{N} of waveforms in the plot (typically 10) * ``baseline'': the
average of the \texttt{N} baselines of the \texttt{N} waveforms, each
one computed with \texttt{extractBaseline()} algorithm * ``RMS'': the
average of the \texttt{N} RMS on the baseline of the \texttt{N}
waveforms; this is a quantification of the average noise * ``maximum'':
for each waveform, the largest value is found; the \emph{maximum} is the
average of the distribution of maxima from the \texttt{N} waveforms, and
the uncertainty is its error * ``peak'': for each waveform, the positive
and negative peaks are found over the baseline, and the largest is used;
the \emph{peak} is the average of the distribution of single waveform
peaks * ``RMS'' is the RMS of that peak value distribution. \\

Additionally, a small statistics library is provided: *
\texttt{MinAccumulator}, \texttt{MaxAccumulator},
\texttt{ExtremeAccumutator}, \texttt{MinAccumulatorN},
\texttt{MaxAccumulatorN} and \texttt{ExtremeAccumutatorN} remember the
\emph{N} largest or smallest (or both) of the values that are
\texttt{add()}'ed to them * \texttt{StatAccumulator} gives average,
error, RMS and related quantites, of the sample given to it element by
element (via \texttt{add()} call) \\

The fundamental analysis algorithms applied on a single waveform are: *
\texttt{findMaximum()}, \texttt{findMinimum()} and
\texttt{findExtremes()} return the position respectively of the maximum,
of the minimum and of both, within the list of values in argument *
\texttt{extractBaselineFromPedestal()} extracts the baseline and its RMS
and uncertainty, as average of the samples in the start of the waveform
* \texttt{extractBaseline()} extracts the baseline of the waveform and
its RMS and uncertainty, as described in the previous section *
\texttt{extractPeaks()} extracts the value and location of the two peaks
of the waveform, with uncertainties, as described in the previous
section * \texttt{extractStatistics()} runs many of the previous
algorithms, and also select the absolute peak.
