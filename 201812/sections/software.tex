\section{Test software}
\label{sec:software}


\subsection{Data sets}
\label{ssec:software:datasets}

The original acquisition software grants the operator some options that affect
the name of the waveform files. An acquisition-driving code was developed that
reduces those options and enforces a file name pattern:
\begin{verbatim}
waveform_CH<channel>_CHIMNEY_<chimney>_CONN_<cable>_POS_<position>_<index>.csv
\end{verbatim}
where the elements, defined in \cref{ssec:labelling}, are:
\begin{description}
  \item[\texttt{<channel>}] is the \emph{oscilloscope channel} the waveform is read from, which ranges from \texttt{1} to \texttt{4}
  \item[\texttt{<chimney>}] is in the form \texttt{<module><side><row>}, \texttt{WE07}
  \item[\texttt{<cable>}] is the label of the cable plugged in the test box, e.g.\ \texttt{V05}
  \item[\texttt{<position>}] is the setting of the test box switch
  \item[\texttt{<index>}] is a waveform counter;
    following Sergi's convention, this is a sequential number unique
    within waveforms on the same connector and oscilloscope channel.
    Assuming that 10 waveforms are taken per channel, position
    \texttt{POS\_1} will have indices from \texttt{1} to \texttt{10},
    \texttt{POS\_2} from \texttt{11} to \texttt{20}, and so on, up to
    \texttt{POS\_8} with indices \texttt{71} to \texttt{80}.
\end{description}

320 waveform files are expected per cable (10 waveforms for each of
32 channels, split in 4 oscilloscope channels and 8 test box switch
positions), and 5760 per chimney. These files are stored in separate
directories, one per chimney. The name of the directory follows the same
convention as above, that is \texttt{CHIMNEY\_<chimney>} (e.g.\ 
\texttt{CHIMNEY\_WE07}).
\\
The whole data set taken in the test session of August 2018 is currently
stored in Fermilab dCache at the path:
\begin{verbatim}
/pnfs/icarus/persistent/commissioning/connectivity/201808
\end{verbatim}



\subsection{Data format}
\label{ssec:software:dataformats}

Each waveform is stored in its own comma-separated value file (CSV). The file name follows the pattern described above. Each line in the file corresponding to a sample, totaling 10000 lines for each waveform. Each sample is described by values in floating point text representation, with scientific notation: time in seconds, and signal level in volt. For example:

\begin{verbatim}
0.0,2.08
1e-07,2.08
2e-07,2.12
[...]
\end{verbatim}

No metadata is saved in the file. This data format can be easily read directly by ROOT \texttt{TGraph}.



\subsection{Streamlined interactive data acquisition}
\label{ssec:software:streamlinedDAQ}

The original code by Sergi Castells required the operator to change
parameters on the command line on each acquisition, plus an additional step at the end of the test each connector, that is to reset a counter stored in a text file.

After a few interactions with this approach, we wrote additional
software (\emph{test driver}) automating most of the steps required from the operator. This enforced a specific procedure both for testing and for software setup.

The procedure comprises testing one chimney at a time, and proceeding from connection 18 (\texttt{V18} or \texttt{S18}) down to 1 (\texttt{V01} or \texttt{S01}), and on each connector test the channel switching the test box to the position from 1 to 8 in strict sequence. The whole sequence must be strictly followed.

More details about the software used for data acquisition and its setup can be found in \ref{appendix-A}.

