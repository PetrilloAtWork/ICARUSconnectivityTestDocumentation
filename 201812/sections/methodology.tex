\section{Test methodology}
\label{sec:methodology}

This test follows broadly the same principles as the one performed in September
2018.
To test the electrical continuity, a test pulse is send, and a response is read,
this time from the connectors to the readout.
\\
We perform two tests, by probing two different paths.
The first path is equivalent to the one used in September 2018: a test pulse is
sent through the test pulse cables at the bottom of the anode frame, where it is
distributed typically to 32 or 256 wires (for the single termination card and
for the eight daisy chained ones, respectively).
The pulse propagates through the wires, the termination card at the top of the
anode frame, the 68 pin cables, the \DBB and finally though the interface
connector on the flange, where it is picked up by our test apparatus.
The path within the \DBB can be appreciated from \cite{ICARUSDBB}, where it
enters through a ``WIRE'' port (\eg \texttt{WIRE9\_L}) and through the
\nanoF{10} capacitor induces onto the front end (\eg \texttt{FE9\_L}).
\\
The second path starts by pulsing a bias voltage inlet.
From there, the pulse is propagated to one side of the nine \DBB (labeled
``Cable AWG20'' in \cite{ICARUSDBB}) through the common and the channel-specific
circuitry before reaching the front end readout.
It should be noted that the bias voltage circuitry is designed for a constant,
large voltage (\V{300}), while the test pulse is small (a few volt) and rapidly
changing.
Therefore, features observed in this test may be not significant for the regular
operation of the detector.
\\
These two paths test different paths of the circuitry.
The first path, pulsing the wire, tests the full signal path, and barely grazes
the \DBB circuitry.
The second one instead skips the wire but goes through the whole set of \DBB
components.
While the flanges have been tested for bias voltage distribution, that test is
not sensitive enough to pin down a failure on a single channel.
Also, as will be described in the result section (\cref{sec:results}), having a
test including the wire and one excluding it allowed to pin down very precisely
a faulty installation where a 68 pin cable was not correctly connected.
\\
Both the paths we use for testing are necessarily \emph{discontiguous}: both
include a \nanoF{10} capacitor on the \DBB just in front of the front end
outlet, and in addition the wire path an uncalibrated, picofarad-level
capacitance on the bottom termination card.
As a consequence, the test can't verify the continuity of the paths, but it has
rather to interpret a response.
\\
In addition, for a few flanges the test on the bias voltage path was performed
also before the flange being mounted on the detector.
That configuration is effectively almost identical to the one on the detector,
with the exception that the wire is not connected at all and, for example, does
not contribute to pick up noise.


\subsection{Test setup}
\label{ssec:methodology:setup}

The test pulse is generated with a ``test box'' designed by Mark Convery
(\cref{fig:TestBox}), which includes a pulse generator and also provides
amplification of the response to the pulse.
After amplification, the responses are digitised by a four-channel oscilloscope
(Tektronix TDS3000C) and sent to a program running on a laptop, which can record
them as waveforms (\cref{fig:TestSetup}).
\begin{figure}
  \caption{\label{fig:TestBox}
    The box used for generating the test pulse and selecting and amplifying the
    response to it.
  }
\end{figure}
\begin{figure}
  \caption{\label{fig:TestSetup}
    The box used for generating the test pulse and selecting and amplifying the
    response to it.
  }
\end{figure}

The path followed by the test pulse is illustrated in the following paragraphs.

The test boxes are the same used in the testing of September 2018, including the
dead channels (channel 2 in one of the boxes, channel 18 in the other), with
some upgrade.
The boxes are powered by three \V{1.5} AA batteries, and they now include a
voltage regulator that stabilizes the output at about \V{3.3}.
The test pulses are emitted at a rate of \Hz{100}, each one a positive square
wave of \V{3.3} with a duration of about \micros{100}.
Each test box includes two outlets for the same pulse.
One is sent directly to the oscilloscope to work as a trigger.
The other is sent to the detector via a lemo to SMA cable: it will be plugged
either to the test pulse cable or to the bias voltage distribution, to implement
one of the two test paths described in \cref{sec:methodology}.
The response to these pulses is read from the front end connectors on the
flange.
The standard setup includes the installation of an empty readout minicrate, its
only purpose to provide mechanical support, and in it a board shaped as a
readout board, which is in effect just an adapter from the front end connector
to a 68 pin cable: each of these ``fake'' readout boards includes two such
adapters, converting at once both the left and the right side connectors,
and allowing for two different 68 pin cables.
At a time, the correct one of the two cables (for example, the left one if the
test is currently pulsing the left bias voltage distribution path) is plugged
back into the test box, conveying the 32 channels from the \DBB side it is
connected to.
The test box contains a eight position switch that selects four contiguous
channels among the 32 from the cable.
Signals from the four channels are amplified and offered each on an independent
outlet.
These four channels are sent to the oscilloscope for digitization and visual
inspection.
The oscilloscope can be driven by commands sent via Ethernet from a laptop,
where a simple data acquisition program drives the readout and recording of the
digitized waveforms.
One of the test boxes also offers ``direct'', non-amplified versions of the
signal.
\\
Noise and grounding have been an issue in the previous session of tests.
Depending on the grounding of the shield wires, they do actually act as shield,
or they rather act as antennas picking up noise.
The upgraded test box allows the option of connecting all the shield wires to
the ground (which is effectively the oscilloscope ground).
This option was regularly used in the December 2018 testing session.
It results into reduced noise and also in reduced signal response, allegedly
because of reduction of cross talk from other cables.
In addition, we grounded the oscilloscope chassis to the detector.


\subsection{Labelling}
\label{ssec:labelling}
