\section{Glossary and abbreviations}
\label{sec:glossary}

\begin{description}
  \item[T300] each of the two modules of the ICARUS detector, including its own cryostat, two drift volumes (TPC), for a reference argon mass of 300 tonnes
  \item[TPC] (\emph{Time Projection Chamber}) is defined as a unit including a single uniform drift volume, a cathode and a (composite) anode. In ICARUS, each cathode is shared by two TPCs. This is also the definition of TPC used in the official simulation and reconstruction software (LArSoft)
  \item[wire] unless specified to be one of the conducting wires within a cable, it's intended to be a TPC wire
  \item[DBB] \emph{Decoupling and Biasing Board}, used to convey bias voltage to the TPC wires and to read the signal induced on those wires
  \item[cable] the 68-wire flat cable connecting 32 TPC wires to a DBB
  \item[channel] a single data acquisition channel, connected to a single TPC wire
  \item[switch position] one of the eight different positions of the test box selecting 4 out of the 32 channels on a cable
  \item[oscilloscope channel] one of the four osciloscope channels used for the digitization of the response to test pulses
%   \item[signal wire] any of the 32 cable wires electrically connected to the TPC wire
%   \item[shield wire] any of the 32 cable wires not electrically connected to the TPC wire (expected to be grounded by the readout board, but sometimes already grounded to the anode frame)
%   \item[BV] bias voltage (and the circuit allowing its distribution to the wires)
%   \item[TC] test capacitance (and the cable allowing to pulse it)
\end{description}
