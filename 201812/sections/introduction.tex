\section{Introduction}
\label{sec:introduction}

ICARUS detector comprises two twin modules, counting more than 50,000
wires\cite{ICARUSTPC}, each one independently read out.
The goal of a ``connectivity test'' is to verify the electric continuity from
each wire to the readout boards. In fact, the aim is to diagnose major problems
early enough that they can be immediately fixed, reduced or mitigated.
In particular, we aim to detect and correct any complete failure on a cable,
which carries 32 contiguous channels, to attempt the correction of failures
involving a few contiguous channels, and to record isolated single channel
failures.

A first test was performed starting on August 2018\cite{ConnTest201808}, where
connectivity was verified between the wires and the cables bringing the signal
to the chimneys. This report describes in detail the test performed on December
2018\footnote{
Part of the data has been eventually collected in February 2019.
}, which extends the test to the decoupling and biasing boards (DBB) and the
interface connector on the flanges. This is part of the wider connectivity test
effort, which is described in \cite{ConnTest201808}.
The relevant detector configuration is described in \cref{sec:detector} and the
test methodology is described in \cref{sec:methodology}.
A summary of some of the findings are reported in \cref{sec:results}.

Details of the operations are described in \cref{sec:operations} and
\cref{sec:software}.
