\section{Operative details}
\label{sec:operations}


The tests subject of this report happened in three distinct sessions:
\begin{description}
  \item[December 2018] the first extensive test including the \DBB and flanges
    took place. All mounted chimneys were eventually covered in three weeks,
    and the data from the test was recorded.
    This excludes \Chimney{WE02}, \Chimney{WE03}, \Chimney{WE04} and
    \Chimney{WW02}, which at that time were not physically available yet.
  \item[February 2019] another session was scheduled and took place in February
    2019, with the goal of performing the test in the final chimney
    configuration; it was split in two subsessions because for half of the
    chimneys optical flanges were going to be installed, and that installation
    could accidentally impact the connections of the other components in the
    chimney. All the chimneys were again tested, but the only recorded data
    was for the few chimneys that were missing it from December 2018.
  \item[March 2019] saw the last session take place, after an extensive
    intervention on all flanges to deposit glue on the clamps holding the
    68-wire cables.
\end{description}


The following sections describe contributions to the test shifts of the
different sessions.
In addition to those, invaluable contributions on many different levels
have been provided by a very large number of people.
Among others, Alberto Braggiotti, Sandro Centro, Angela Fava, Alberto Guglielmi,
Connie Meng, Claudio Montanari, Donatella Torretta, Zachary Williams
and others that we are surely forgetting to list here but who were nonetheless
crucial to the success of the test.


\subsection{December 2018 test}
\label{ssec:operations:December2018}

The session took place between December 1 and 21, 2018.
The goal of the test was to test all channels served by the 72 short chimneys,
plus all the channels from the first induction wire plane. The session fell
short of that goal because of some of the chimneys were not installed by the end
of the test time window.

Both the test through the test capacitance and through the bias voltage feed
were systematically performed, but only the data from the test capacitance test
was recorded, covering the channels from all the tested chimneys.

Resources (documentation material, communication, ...) include:
\begin{itemize}
  \item SBNFD electronic logbook (\href{http://dbweb6.fnal.gov:8080/ECL/sbnfd/E/search?id=&id_from=&id_to=&t_after=12\%2F01\%2F2018&t_before=12\%2F24\%2F2018&tag\%3AReadout+Testing=on&action=Search}{tag: \emph{Readout Testing}});
  \item \href{https://docs.google.com/spreadsheets/d/1zR-Ytg8CJ5gw-f--yKqr1PXTCiJ0MirVgbQ_ybcFCKM/edit?usp=sharing}{shift organization spreadsheet};
  \item mailing list \href{https://listserv.fnal.gov/scripts/wa.exe?A0=ICARUS-TPC-CONNECTIVITYTEST}{icarus-tpc-connectivitytest@listserv.fnal.gov};
  \item collected data is available in dCache
    ({\small\texttt{/pnfs/icarus/persistent/commissioning/connectivity/201812}})
    and BlueArc
    ({\small\texttt{/icarus/data/commissioning/connectivityTest/201812}}) storage
    (including the updates from the following sessions);
  \item software is available in GitHub
    (\href{https://github.com/PetrilloAtWork/ICARUSconnectivityTest}{PetrilloAtWork/ICARUSconnectivityTest}); the software was progressively updated during the test.
\end{itemize}

There were two 3.5-hour shifts per day, which often dragged into 4 hours, and
on each shift the four operators were split into two groups, usually with
two people each, and each group was assigned to one test stand.
The two groups tested chimneys belonging to different TPCs whenever possible to
minimize the interference of the test apparati.
Complete testing of a single chimney took from one to two hours, barred
problems, with an average of roughly 1.5 hours. The move of the test equipment
and its setup, and the transfer of a readout minicrate from a chimney to another
took a relevant part of the test time. The readout crate were mounted by
a small number of trained people.
Check boards were filled marking the problems or noticeable features of each
channel.
The following people participated to the test shifts:
\begin{itemize}
  \item Howard Budd, Tejin Cai, Mehreen Sultana \emph{(University of Rochester)};
  \item Animesh Chatterjee \emph{(University of Texas at Arlington)};
  \item Hang Su \emph{(University of Pittsburgh)};
  \item Biswaranjan Behera, Tyler Boone, Ivan Caro, Chris Hilgenberg, Hannah Rogers \emph{(Colorado State University)};
  \item Mark Convery, Laura Domine, Gianluca Petrillo, Kazuhiro Terao, Yun-Tse Tsai \emph{(SLAC National Accelerator Laboratory)}.
\end{itemize}



\subsection{February 2019 test}
\label{ssec:operations:February2019}

The session took place between February 18 and March 1, 2019.
The goal of the test was to retest all channels after two months of installation
operations, to verify that the results are stable, and to collect data for the
chimneys that have been mounted or replaced since the previous session.
Only the test through the test capacitance was performed, and again the channels
on the shorter wires served by the bottom flange of the tall chimneys was not
tested.

Resources (documentation material, communication, ...) include:
\begin{itemize}
  \item SBNFD electronic logbook
    (\href{http://dbweb6.fnal.gov:8080/ECL/sbnfd/E/search?id=&id_from=&id_to=&t_after=02\%2F18\%2F2019&t_before=03\%2F01\%2F2019&tag\%3AReadout+Testing=on&action=Search}{tag: \emph{Readout Testing}});
  \item \href{https://docs.google.com/spreadsheets/d/1wwkhF9-X4gV3Hmp61LN8EbjIY5xaMLv4yKINKjvRKVQ/edit?usp=sharing}{shift organization spreadsheet};
  \item mailing list \href{https://listserv.fnal.gov/scripts/wa.exe?A0=ICARUS-TPC-CONNECTIVITYTEST}{icarus-tpc-connectivitytest@listserv.fnal.gov}
    (same as on December 2018);
  \item updated data is available in the same location as December 2018 data.
\end{itemize}

The shifts developed in the same way as in the previous session.
After the installation of the decking on top of the red box structure, the
transition from one chimney to the next was noticeably easier and faster.
The availability of backup readout minicrates and of additional SMA cables also
contributed to speed up the testing by pipelining their mounting.
The following people participated to the test shifts:
\begin{itemize}
  \item Howard Budd, Ryan Howell \emph{(University of Rochester)};
  \item Animesh Chatterjee, Zachary Williams \emph{(University of Texas at Arlington)};
  \item Biswaranjan Behera, Chris Hilgenberg \emph{(Colorado State University)};
  \item Mark Convery, Gianluca Petrillo, Yun-Tse Tsai \emph{(SLAC National Accelerator Laboratory)}.
\end{itemize}


\subsection{March 2019 test}
\label{ssec:operations:March2019}

The session took place between March 4 and March 8, 2019.
The further test was necessary because of interventions inside the chimneys
following the last test, and in particular because of the application of glue
on the clamps of the 68-pin cables.

The goal of the test was to retest all channels and verify that the last
interventions had not affected the connectivity, and to have a final survey of
the status of the channels. The coverage of the test was the same as for the
previous session.

Operatively, the test developed quite differently from the last time.
The focus was on quick testing of all the chimneys, because of strict time
constraints on the sealing of the flanges and because no new problem was
expected, and only the test through the test capacitance was performed, with no
data recording.
Based on the previous experiences, a single team was operating on the detector,
with one operator moving the test board, one browsing through the channels and
one taking notes. For the best consistency, the ``decision'' on the goodness of
a channel was taken always by the same operator. A fourth operator was in charge
of setting up the test for the next chimney while the rest of the crew was
testing. A full test equipment set was available for a fifth operator to perform
targeted tests, until the breakage of one of the test boxes.
The shifts were therefore full day for some of the shifters, and it was possible
to complete almost one row of short chimneys per day.

Resources (documentation material, communication, ...) include:
\begin{itemize}
  \item SBNFD electronic logbook (\href{http://dbweb6.fnal.gov:8080/ECL/sbnfd/E/search?id=&id_from=&id_to=&t_after=03\%2F02\%2F2019&t_before=03\%2F15\%2F2019&tag\%3AReadout+Testing=on&action=Search}{tag: \emph{Readout Testing}});
  \item \href{https://docs.google.com/spreadsheets/d/1wwkhF9-X4gV3Hmp61LN8EbjIY5xaMLv4yKINKjvRKVQ/edit?usp=sharing}{shift organization spreadsheet}
    (same as for February 2019);
  \item mailing list \href{https://listserv.fnal.gov/scripts/wa.exe?A0=ICARUS-TPC-CONNECTIVITYTEST}{icarus-tpc-connectivitytest@listserv.fnal.gov}
    (same as on December 2018);
\end{itemize}

The following people participated to the test shifts:
\begin{itemize}
  \item Howard Budd, Ryan Howell \emph{(University of Rochester)};
  \item Biswaranjan Behera \emph{(Colorado State University)};
  \item Mark Convery, Gianluca Petrillo, Yun-Tse Tsai \emph{(SLAC National Accelerator Laboratory)}.
\end{itemize}
