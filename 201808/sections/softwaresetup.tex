\section{Setup of acquisition code}
\label{app:DAQsoftwareSetup}

The data acquisition pattern followed in this test has been described in
\cref{ssec:streamlined-interactive-data-acquisition}.
In this section, technical details are described for the set up of the software
for that acquisition.

All the original code written by Sergi Castells and the following extensions are
published in a public GIT repository. A copy of that repository is currently
available in GitHub as
\href{https://github.com/PetrilloAtWork/ICARUSconnectivityTest}{\texttt{PetrilloAtWork/ICARUSconnectivityTest}}\footnote{%
URL: \url{https://github.com/PetrilloAtWork/ICARUSconnectivityTest}.}.
The software used for data acquisition in September 2018 was being developed as
the need arose. At the beginning, Sergi's code, roughly equivalent to the tag
\texttt{v01\_00} of the repository above, was used
directly. By the end of the data acquisition, version \texttt{v03\_00} was used,
and code was drawn from that version for the analysis as well.
In the following connectivity test, planned for December 2018, the starting
version will be \texttt{v04\_00} (see \cref{apx:DAQsoftwareSetup:Future}).

The test driver is unfortunately hard to correctly set up the first time.
The software is all written in python 2, and it requires both the python
Visual Instrument Software Architecture (PyVISA, exposing the module
\texttt{visa}) and CERN ROOT (PyROOT, exposing the module \texttt{ROOT})
interfaces. It proved hard to get to a configuration where python would
load both at the same time on OSX, while no Linux laptop was tried. The
pattern that seemed more likely to succeed was to install PyVISA
following online instructions found at
\url{https://pyvisa.readthedocs.io/en/master/index.html}, which
included installing National Instrument backend software\footnote{%
Registration to National Instruments web site is required in
order to download the software, but no fee payment was required to
download the driver package.
}, that we used in its release 18.0. After having tested that the \texttt{visa}
module correctly work, again as described in the linked documentation,
we compiled ROOT (6.14) from source \emph{in the same environment} (that
is, in the same shell).

The setup of the test instrumentation is the same as for the software
from Sergi. The data acquisition session is run within a single
\texttt{python} session. Its setup includes importing the test driver
module (\texttt{testDriver.py}), setting the IP address of the
oscilloscope in use, and finally instantiate the test driver. A possible
setup sequence is:
\begin{verbatim}
from testDriver import *
loadScopeReader('192.168.230.29')
reader = ChimneyReader()
\end{verbatim}
After this per-session setup, the data acquisition is driven by choosing
the chimney (e.g. \texttt{reader.start("EW09")}) and repeatedly call the
acquisition function (\texttt{reader.next()}). It is also possible to
skip to remove the last acquisition and repeat it, or skip to a
different one.

Note that for up-to-date instructions the file \texttt{README.md} in the GIT
repository supersedes the information provided here.


\subsection{Improvement implemented for December 2018 test}
\label{apx:DAQsoftwareSetup:Future}

This section shortly lists improvements that are implemented and will be tested
for the following connectivity test, the last without the official ICARUS DAQ,
planned for December 2018.
\begin{itemize}
  \item single script working for all oscilloscopes
  \item configuration of the script driven by configuration file(s)
  \item integrated verification of the correctness of the data format at the end
        of the acquisition from a chimney
  \item generation of a data transfer script
  \item single connection to the oscilloscope for the whole session (more error
        prone but \emph{much} faster; \texttt{ChimneyReader.scope().reconnect()}
        may help to recover connection issues
  \item ability to jump to a chosen cable and position at any time
        (\texttt{ChimneyReader.jumpTo()})
\end{itemize}
The expected acquisition flow is outlined as follows:
\begin{enumerate}
  \item preparation of the data acquisition: start of the python interactive
        shell, initialisation of `testDriver` helper object with the proper
        configuration file, selection of the first chimney to read out;
  \item sequential acquisition of waveforms from all cables, with the same
        pattern as in September 2018;
  \item after completion, verification of the data written on the local disk;
        this may take 5-10 minutes, while the operators may ``close'' the tested
        chimney and prepare for the acquisition of the next one (but note that
        if data is found incomplete, more acquisition might need to take place);
  \item after data is validated, generation of a data transfer script
  \item execution of the generated script into a separate shell, going in
        parallel with the rest of the activity
\end{enumerate}
The remote destination of the data is specified in the configuration file.
